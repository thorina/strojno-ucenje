\documentclass[conference]{IEEEtran}
\usepackage[croatian]{babel}
\usepackage{url}
\hyphenation{op-tical net-works semi-conduc-tor}
\usepackage{amsmath}
\begin{document}
\title{Ekstrakcija likova iz kratkih pri\v{c}a}
\author{\IEEEauthorblockN{Gorana Leva\v{c}i\'c}
\and
\IEEEauthorblockN{Tomislav Horina}}
\maketitle

\begin{abstract}
U ovom radu smo radili klasifikator za likove u kratkim pri\v{c}ama. Koristili smo kratke pri\v{c}e s Gutenberg projecta.Pri\v{c}e smostalno ozna\v{c}avali jer postoje\'ca rije\v{s}enja ne prepoznaju na primjer kralja kao lika.Promatrali smo pona\v{s}anje klasifikanirao na 4 na\v{c}ina zadavanja tesnog skupa.Vrste su :\begin{enumerate}
\item bez ikakve promjene
\item pri\v{c}e bez interpukcijski znakova
\item sve rije\v{c}i u lowercaseu
\item sve rije\v{c}i u lowercaseu i bez interpukcijski znakova
\end{enumerate} 
Isprobali smo algoritme klasifikacije koji su preporu\v{c}eni za izradu tog modela -- skriveni Markovljev model(eng. Hidden Markov Model, HMM) i uvjetno slu\v{c}ajno polje(eng. Conditional Random Fields, CRF) te smo Stanford Ner u\v{c}ili na na\v{s}e ozanke.
(sada tu treba ici nesto  rezzultatima to treba nadopisati.)     
\end{abstract}
\section{Uvod}
U ovom projektu bavit \'cemo se ekstrahiranjem likova iz kratkih pri\v{c}a, konkretno pri\v{c}a za djecu. Taj problem pripada problemu ekstrakcije, odnosno identifikacije entiteta u tekstu, poznatiji pod engleskim nazivom named-entity recognition (NER). NER je podvrsta zada\'ce crpljenja obavijesti (information
extraction), u kojoj se svakom elementu teksta pridjeljuje neki atribut. U op\'cem
slu\v{c}aju imamo vi\v{s}e atributa, na primjer osoba, lokacija, vrijeme, iznos novca i
drugi, te vi\v{s}e rije\v{c}i mo\v{z}e \v{c}initi entitet kojem se pridjeljuje jedan atribut. Jasnije
je iz sljede\'ceg primjera:\\
Jim bought 300 shares of Acme Corp. in 2006.\\
$|Jim|_{person}$ bought 300 shares of $|Acme Corp.|_{organization}$ in $|2006|_{time}$.\\
U na\v{s}em slu\v{c}aju imamo samo jedan atributa, lik, koji odre\dj{}uje tko su sve likovi u pri\v{c}i.
\section{Podaci}
Tekstove pri\v{c}a smo skidali s stranice Project Gutenberg. Potom smo samostalno ozna\v{c}avali likove. To smo radili po\v{s}to postoje\'ca rije\v{s}enja ne prepoznaju, na primjer kralj, kraljica, princ i tako dalje, kao likove. Svakoj rije\v{c}i smo dodljeljivali klasu O ,\v{s}to ozna\v{c}ava $"Other"$, i klasu C ,\v{s}to ozna\v{c}ava $"Charachter"$.

U na\v{s}im modelima gledat \'cemo dali na njega utje\v{c}u veli\v{c}ine slova i postojanje interpunkcijski znakova, to jest imat \'cemo \v{c}etri vrste tesnog skupa: 
\begin{enumerate}
\item bez ikakve promjene
\item pri\v{c}e bez interpukcijski znakova
\item sve rije\v{c}i u lowercaseu
\item sve rije\v{c}i u lowercaseu i bez interpukcijski znakova
\end{enumerate} 


\section{Opis kori\v{s}tenih meotda}
Koristili smo algoritme klasifikacije koji su preporu\v{c}eni za izradu tog modela :
\begin{enumerate}
\item Skriveni Markovljev model(eng. Hidden Markov Model, HMM)
\item Uvjetno slu\v{c}ajno polje(eng. Conditional Random Fields, CRF)
\end{enumerate}
Dodatno smo u\v{c}ili Stanford Ner na na\v{s}e ozanke.

\subsection{Skriveni Markovljev model}
Skriveni Markovljev model (HMM) prvog reda jest skup slu\v{c}ajnih varijabli koji se sastoji od dva podskupa, $Q$ i $O$ :
\begin{itemize}
\item $Q = Q_1, \ldots , Q_N$ -- skup slu\v{c}ajnih varijabli koje poprimaju diskretne vrijednosti
\item $O = O_1, \ldots , O_N$ -- skup slu\v{c}ajnih varijabli koje poprimaju diskretne ili kontinuirane vrijednosti.
\end{itemize}
Te varijable zadovoljajvaju sljede\'ce uvjete:
\begin{enumerate}
\item \label{prvi_uvjet} $P(Q_t|Q_{t-1},O_{t-1}, \ldots, Q_{1},O_{1}) = P ( Q_{t}|Q_{t-1})$ (\ref{prvi_uvjet})
\item \label{drugi_uvjet} $P(O_t|Q_T,O_T, \ldots, Q_{t+1},O_{t+1},Q_t,Q_{t-1},O_{t-1}, \ldots  Q_{1},O_{1}) = P ( O_{t}|Q_{t})$  (\ref{drugi_uvjet})
\end{enumerate}


Relacija (\ref{prvi_uvjet}) ka\v{z}e da je vjerojatnost da se, za neko $ t \in \{1,2,\ldots , N \}$, nalazimo u stanju $Q_t$ uz uvjet da su se dogodila prethodna stanja $Q_1, \ldots, Q_{t-1} $ i da su emitirani simboli $O_1, \ldots , O_{t-1} $ jednaka tranzicijskoj vjerojatnosti iz stanja $Q_{t-1}$ u stanje $Q_t$.


Relacija (\ref{drugi_uvjet}) povla\v{c}i da realizacija nekog opa\v{z}anja u sada\v{s}njem stanju ovisi samo o tom stanju . Vjerojatnosti iz relacije (\ref{drugi_uvjet}) nazivamo emisijske vjerojatnosti i ka\v{z}emo da stanje $Q_t$ emitira simbol $O_t$. 

Skriveni Markovljev model zadan je sljede\'cim parametrima:
\begin{itemize}
\item $N$ -- broj stanja u kojima se proces mo\v{z}e nalaziti, $ S = \{1, \ldots, N\}$ , $S$ -- kup svih stanja procesa
\item $M$ -- broj mogu\'cih opa\v{z}anja ,$B = \{ b_1, \ldots, b_M\}$, $B$ -- skup svih opa\v{z}enih vrijednosti
\item $L$ -- duljina opa\v{z}enog niza ,  $X=(x_1,\ldots , x_L)$, $X$ -- opa\v{z}eni niz 
\item $A$ -- matrica tranziciskih vjerojatnosti , $ A = \{a_{ij}\}, a_{ij} = P(Q_{t+1} = j | Q_t = i ) , 1 \leq i,j \leq N$
\item $E$ -- matrica emisijskih vjerojatnosti $E\{ e_j(k), e_j(k) = P ( O_t = b_k | Q_t = j) , 1 \leq j \leq N , 1 \leq k \leq M$ 
\end{itemize}
\subsection{Uvjetno slu\v{c}ajno polje}
Uvjetna slu\v{c}ajna polja (CFR) je diskriminativni vjerojatnosni model strojnog u\v{c}enja za struktuiranu predikciju koja je temeljena  na modelima neusmjerenih grafova. Neka je dan vektor $x = \{ x_1 , x_2, \ldots , x_T\}$ gdje je $T$ du\v{z}ina sekvence, a svaki $x_i$ predstavlja vektor karakteristika podataka na poziciji $i$. Za svaki od tih vektora porebno je odrediti oznaku (tag) $y_i$. U slu\v{c}aju odre\dj{}evianja vrste rije\v{c}i, $T$ bi bio broj rije\v{c}ima u tekstu, $y_i$ vrsta rije\v{c}i  na poziciji $i$ , a za svaki $x_i$ bi bio vektor informacija o $i$-toj rije\v{c}i   kojima se raspola\v{z}e sama rije\v{c}, informacije o prefiksima, sufiksima i veli\v{c}ini slova \ldots. 

% mozda za obrisati jer nemamo lineani model 
Neka su $x$ i $Y$ slu\v{c}ajne varijable, $ w = \{w_k\} $ realni vektor parametara i $F = \{ f_k(y_t y_{t-q},x_t)\}_{k=1}$ skup realnih karakteristi\v{c}nih funkcija. 
Linearni uvjeta slu\v{c}ajna polja definiraju se uvjetnom raspodjelom 
\[
P(y|x) = \frac{1}{Z(x)} \prod_{t=1}^T exp \left[ \sum_{k=1}^K w_k f_k(y_t,y_{t-q},x_t)\right] 
\]
pri \v{c}emu je \[ Z(x) = \sum_y \prod_{t=1}^T exp \left[ \sum_{k=1}^K w_k f_k(y_t,y_{t-q},x_t)\right]  \]
Karakteristi\v{c}na funkcija mo\v{z}e se promatrati proizvoljne karakteristike 
sekvence , tako da se vektor $x_t$, zapravo mo\v{z}e zamijneiti cjelokupnom sekvencom opservacija $x$.
Linearna uvjetna slu\v{c}ajna polja omogu\'cavaju promatranje samo dva uzastopna taga, zbog \v{c}ega se model mo\v{z}e posmatrati kao lanac. Linarna uvjetna slu\v{c}ajna polja mogu se definirati i preko grafa nad skupom \v{c}vorova $U=X \cup Y$ kao  
\begin{align*}
P(y|x) &= \frac{1}{Z(x)} \prod _{t=1}^T \Psi_t (y_t, y_{t-1},x_t)  \\
\Psi_t(y_t,y_{t-1},x_t) &= exp \left[ \sum_{k=1}^K w_k f_k(y_t,y_{t-q},x_t)\right]
\end{align*}
% generalna CRF
Neka je $G$ faktor graf nad slu\v{c}ajnim varijablama $X$ i $y$. $(X,Y)$ je uvjentno slu\v{c}ajno polje ako se za neke vrijednosti $x$ varijabli $X$, uvjetna vjerojatnost $P(y|x)$ 
% faktoriše prema G
faktorira prema faktoru $G$.  
Svako uvjetno slu\v{c}ajno polje se faktoria prema nekom faktoru grafa G.  Dati faktor graf $G$ i njegov skup faktora $\{\Psi_a \}$ definiraju raspodjelu \[ P(y|x) = \frac{1}{Z(x)} \prod_{i=1}^T \Psi_a(x_a,y_a) \]
gdje je \[ \Psi_a(x_a,y_a) = exp \left[ \sum_{k=1}^K w_{ak} f_{ak}(x_a,y_a)\right] \]
gdje u indeksiranju karakterisičnih funkcija i odgovaraju\'cih parametara sudjeluje i oznaka faktora $a$ jer svaki fakotr mo\v{z}e da ima svoj skup parametara. Faktor $\Psi_a$ zavisi od unije nekih podskupova $X_a \subseteq X$ i $Y_a \subseteq Y$. Uslu\v{c}aju linearnih uvjetnih slu\v{c}ajnih polja, skup $Y_a$ je mogao da sadr\v{z}i samo dva susjedna taga. Skup svih parametara mo\v{z}e se podjeliti na klase $C=\{C_1, C_2, \ldots , C_p\}$ gdje svaka klasa $C_p$ koristi isti skup karakteristi\v{c}nih funkcija $\{ f_{pk} 8 x_c,y_c)\}_{k= 1,\ldots,K(p)}$ i vektor parametara $w_{kp}$ veli\v{c}ine $K(p)$. Raspodjela vjerojatnosti se mo\v{z}e napisati kao 
\begin{align*}
P(y|x) &= \frac{1}{Z(x)} \prod_{i=1}^T \Psi_a(x_a,y_a) \\
\Psi_c(x_c,y_c;w_p) &= exp \left[ \sum_{k=1}^K w_{pk} f_{pk}(x_c,y_c)\right]
\end{align*}
 
\subsection{Stanford Ner}
\section{Rezultati}
hmm nevalja 


\begin{thebibliography}{1}
\bibitem{IEEEhowto:Rudman}
M.~Rudman, \emph{Kompleksnost skrivenih Markovljevih modela}, (diplomski rad).\hskip 1em plus
  0.5em minus 0.4em\relax Prirodoslovno-matematički fakultet, 2014.
  
  
\bibitem{IEEEhowto:Medic}
I.~Medic, \emph{Uslovna slu\v{c}ajna polja}, (master rad).\hskip 1em plus
  0.5em minus 0.4em\relax  Univerzitet u Beogradu, Matemati\v{c}ki fakultet, 2013
\end{thebibliography}  


\end{document}


